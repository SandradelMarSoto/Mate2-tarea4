\documentclass{article}
    \usepackage[utf8]{inputenc}
    \usepackage[spanish]{babel}
    \decimalpoint
    \usepackage{amsmath, amssymb, amsfonts}
    \usepackage{xparse, physics}
    \usepackage{cancel}
    \usepackage{graphicx}
    \usepackage[usenames]{color}
    \parindent  = 0mm
    \parskip    = 4mm
    \usepackage[text={20cm,25cm},centering,top=1.5cm,bottom=1.5cm,letterpaper,showframe=false]{geometry}

    \definecolor{azul}{RGB}{10,80,190}
    \definecolor{verde}{RGB}{0,120,50}
    \definecolor{rojo}{RGB}{190,80,10}
    \renewcommand{\baselinestretch}{1.2}

    \begin{document}
        \title{Tarea/Examen 4 - Aplicaciones}
        \author{Careaga Carrillo Juan Manuel \\ Quiróz Castañeda Edgar \\ Soto Corderi Sandra del Mar}
        \date{Miércoles 1 de junio de 2018}
        \maketitle

        \begin{enumerate}
            % Ejercicio 1
            \item {
                Encontrar puntos máximos, mínimos y sillas de $f(x,y)=(x^2+y^2)e^{(y^2-x^2)}$.
                Graficar la función.

                \color{azul}
            }
            % Ejercicio 2
            \item {
                Encontrar máximos y mínimos de $f(x,y,z)=x^2y^2z^2$ con la restricción
                $x^2+y^2+z^2=1$.

                \color{azul}
            }
            % Ejercicio 3
            \item {
                Encontrar los valores máximos y mínimos absolutos para $f(x,y)=xy^2$ sobre
                la región $D=\{(x,y)|x\geq 0, y\geq 0 \text{ y } x^2+y^2\leq 3\}$.

                \color{azul}
            }
            % Ejercicio 4
            \item {
                Encontrar los puntos sobre el cono $z^2=x^2+y^2$ más cercanos al punto
                $(4,2,0)$.

                \color{azul}
            }
            % Ejercicio 5
            \item {
                Encontrar tres números positivos que sumen $12$ y cuya suma de sus cuadrados
                sea lo más pequeña posible.

                \color{azul}
            }
            % Ejercicio 6
            \item {
                Tres alelos, $A$, $B$ y $O$ determinan los 4 tipos de sangre, a saber
                $A(AA \text{ o } AO)$, $B(BB \text{ o } BO)$, $O(OO)$ y $AB$. La ley
                de Hardy-Weinberg establece que la proporción de individuos de una
                población que llevan dos alelos diferentes es
                \[ P=2pq+2pr+2rq \]
                donde $p$, $q$ y $r$ son las proporciones de $A$, $B$ y $O$ respectivamente
                en la población. Usar el hecho de que $p+q+r=1$ para demostrar que $P$
                es cuando mucho $\frac{2}{3}$.

                \color{azul}
            }
            % Ejercicio 7
            \item {
                Reducir las ecuaciones a la forma canónica, clasificarlas, analizarlas
                usando trazas y otros planos, y bosquejar las gráficas:
                \begin{enumerate}
                    % a)
                    \item{
                        $x^2+2y-2z^2=0$
                    }
                    % b)
                    \item{
                        $x^2-y^2+z^2-2x+2y+4z+2=0$
                    }
                \end{enumerate}
                La traza es la intersección de la supercie con los planos coordenados
                $XY,YZ \text{ y } XZ$.

                \color{azul}
            }
        \end{enumerate}
    \end{document}